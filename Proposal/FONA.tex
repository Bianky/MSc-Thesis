
\documentclass[11pt]{article}
\usepackage[utf8]{inputenc}
\usepackage[a4paper, total={165mm, 216mm}]{geometry}

% package for language setting ---------------------------------------------
\usepackage[english]{babel}

% package used for paragraph formatting ------------------------------------
\usepackage{parskip, fancyhdr, url, rotating, appendix}

% packages used for images -------------------------------------------------
\usepackage{graphicx}
\graphicspath{{images/}}
\usepackage{float}

% packages used for references ---------------------------------------------
\usepackage{xcolor} % Load this for color definitions
\usepackage{natbib}
\usepackage{hyperref}
\hypersetup{
    colorlinks = true,
    linkcolor = gray,
    citecolor = gray,
    urlcolor  = gray
}
\bibliographystyle{apalike}


% package used for degree symbol -------------------------------------------
\usepackage{textcomp, gensymb}


% packages used for tables -------------------------------------------------
\usepackage{multirow, makecell, booktabs,longtable,array,ragged2e, graphicx, chngpage, afterpage, geometry, color}

% package used for caption -----------------------------------------------
\usepackage[format=hang,font=small,labelfont=bf]{caption}
\captionsetup{justification   = raggedright,
              singlelinecheck = false}
              

% no indentation
\setlength\parindent{0pt}


\begin{document}

\section*{Introduction}

Tropical forests are critical ecosystems that support vast biodiversity, provide vital ecological services, and sustain the livelihoods of millions of people \citep{bormaCarbonContributionsSouth2022, mulatuBiodiversityMonitoringChanging2017}. However, over half of these forests have been deforested and converted into agricultural or urban land, resulting in the loss of essential ecosystem services \citep{chazdonNaturalRegenerationTool2016}. Natural forest regeneration, particularly secondary succession, is a promising solution to restore these ecosystems. Secondary succession is the process through which native species, such as trees, recolonize disturbed land, benefiting from existing soils and propagules \citep{poorterSuccessionalTheories2023}. However, this recovery is influenced by several landscape factors such as forest age, cover, and connectivity, which can regulate seed availability, diversity, and dispersal. Despite this, there is limited research on how these factors influence seed dispersal and regeneration dynamics, particularly comparing dry and wet tropical forests.

This study aims to investigate the impact of forest landscape factors—such as forest age, cover, and connectivity—on seed diversity, abundance, and dispersal mode in dry and wet tropical secondary forests during early succession. The research will focus on understanding how these landscape characteristics influence seed dynamics, which are fundamental to the regeneration process. Additionally, the study seeks to compare the differences in seed dispersal mechanisms and diversity between dry and wet tropical forest ecosystems.

It is hypothesized that seed richness and evenness will increase with forest age in both dry and wet tropical forests, as early successional species are gradually replaced by later successional species over time \citep{chazdonNaturalRegenerationTool2016}. It is further hypothesized that forest connectivity and cover will have a negative impact on seed biotic dispersal in dry tropical forests due to the higher occurrence of abiotic dispersal in these areas. In contrast, in wet tropical forests, forest connectivity and cover are expected to positively impact biotic dispersal, as biotic dispersal mechanisms are more dominant. Additionally, it is expected that forest connectivity and cover will have a positive effect on seed richness and diversity in wet tropical forests, as biotic dispersal increases the potential for seed movement and establishment \citep{dentUnitingNicheDifferentiation2021}.

\section*{Methodology}
This research will be carried out in dry tropical forest located in Nizanda (16°39′ N, 95°00′ W), Mexican province of Oxaca and in the wet tropical forest located in Loma Bonita (16°01’ N, 90°55’), Mexican province of Chiapas. The dry tropical forest has a mean annual temperature of 27.7°C with a mean annual precipitation of 900mm. It is characterized as dry tropical deciduous forest \citep{hordijkLandUseHistory2024}. The wet forest has a mean annual temperature of 24°C and a mean annual precipitation of 3000mm. The vegetation varies between semideciduous and lowland tropical rain forest \citep{hordijkLandUseHistory2024}.

There is 13 (25m x 25m) plots established in the dry tropical forest and 20 (25m x 25m) in the wet tropical forest. These plots were established on an recently abandoned pasture or agricultural land by the PANTROP project in 2020 \citep{hordijkLandUseHistory2024}.

This research will be composed of a fieldwork and a spatial analysis. The fieldwork will serve for data collection on the seed variables, and the spatial analysis will be a mean to acquire insights into the forest landscape variables. 
Additionally, the PANTROP database will be used to access historical data from previous years. 

For this research, a fieldwork will be conducted in the dry tropical forest in Nizanda, Mexico. During the fieldwork, litter traps containing plant litter and seeds will be collected. This litter will be dried, weighted and sorted into leaves, twigs, reproductive tissue and others. 
Subsequently, seed counting and identification will be performed with a seed catalogue provided by UNAM in the laboratory in Morelia. The seeds will be classified into life form groups such as herbs, grasses and trees. Subsequentially, absed on literature, their mode of dispersal will be identified.

Unfortunately, it is not possible to conduct fieldwork in the wet tropical forest in Loma Bonita due to the political situation at place. However, a data on wet tropical forest will be retrieved from the PANTROP project database to be able to analyse and compare wet and dry tropical forest.

Landsat satellite imagery will be used for data acquiry on the forest landscape factors. Namely, forest age, forest cover and forest connectivity. Additionally, the forest landscape factors will be calculated using a time-series algorithm called AVOCADO and R packages landscapemetric and lconnect.

To investigate the relationship Linear Mixed Model will be performed on dependent variable: seed richness, seed evenness and seed dispersal mode and the independent variables: forest age, cover and forest connectivity. The model will be evaluated using R2, RMSE and bias.

\newpage
\bibliography{references}

\end{document}