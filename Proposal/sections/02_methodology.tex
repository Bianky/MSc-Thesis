
% METHODOLOGY
% •	Study area description, if any
% •	Justify certain choices of methods and datasets
% •	Be clear on what will be investigated using what metrics. For example, the research will study the uptake of solar panel installation for different provinces by comparing the number of households that have solar panel installations since 2015. 
% •	Be clear on how to validate the results and using what metrics. For example, assess the accuracy of detecting solar panels using overall accuracy, user’s accuracy, precision, etc.,
% •	Clearly describe any ground truth or reference data, e.g., how to access and how/what to collect
% •	page indication: 2 

\subsection{Study sites}
This research will be carried out in dry tropical forest at an elevation of 90-120~m a.s.l. located in the area of Nizanda (16°39' N, 95°00' W), Mexican province of Oaxaca and in the wet tropical forest at an elevation of 115-300~m a.s.l. located in the area of Loma Bonita (16°01' N, 90°55'), Mexican province of Chiapas. 

The dry tropical forest has a mean annual temperature of 27.7°C and seasonal rainfall with a mean annual precipitation of 900~mm where 90\% of the total annual rainfall occurs between May and October. It is characterized as dry tropical deciduous forest \citep{hordijkLandUseHistory2024}. 

The wet forest has a mean annual temperature of 24°C and a mean annual precipitation of 3000~mm with a relative dry season between February and April accounting for less than 10\% of the total annual rainfall. The vegetation varies between semideciduous and lowland tropical rain forest \citep{hordijkLandUseHistory2024}.

There are twenty (25~m x 25~m) plots established in both the dry and wet tropical forest. These plots were established on recently abandoned (0-10 months following abandonment) pasture or agricultural land by the PANTROP project in March 2020 \citep{hordijkLandUseHistory2024}. If needed, plots were fenced off to prevent grazing. 


\subsection{Seed attributes}
To evaluate the seed diversity and dispersal mode in both forest sites (wet and dry), four seed traps were established in each plot. In the dry forest, the seed traps were rectangular with the dimensions of 50~cm x 50~cm and in the wet tropical forest, the seed traps were circular with the diameter of 1~m. The seed traps were positioned 6~m from the plot's edge in each corner of the plots. The seeds were collected monthly for a duration of one year in 2023. The content of the four traps was combined into one sample per plot.

Subsequently, the collected seeds were separated into groups of unique species. Each group of unique seed species  was counted by the number of individuals, weighted by the total weight per group and identified by species. The identification of seed species was performed with a seed catalog provided by Universidad Nacional Autonóma de México (UNAM) and with the support of research assistant Marina Beatriz Hernandez Mendez, an expert in tropical wet forest seed identification, and ..., an expert in tropical dry forest seed identification. 

% define how to calculate the seed attributes?
In order to evaluate seed richness, evenness and dispersal mode. The mode of dispersal is based on literature.

% For this research, a fieldwork will be conducted in the dry tropical forest in Nizanda, Mexico. 
% Unfortunately, it is not possible to conduct fieldwork in the wet tropical forest in Loma Bonita due to the political situation at place. However, data on wet tropical forest and historical data on both sites will be retrieved from the PANTROP project database to be able to analyse and compare wet and dry tropical forest.

\subsection{Forest landscape factors}
To evaluate the forest landscape factors, remote sensing techniques will be employed, as historical field data is scarce and interview-based data can be unreliable and difficult to obtain \citep{decuyperContinuousMonitoringForest2022}. In this context, the forest landscape factors are defined as follows: 1) \textbf{forest age}: the number of years a forest remained undisturbed, up to a maximum of 25 years (the limit of satellite imagery availability), 2) \textbf{forest cover}: the percentage of surrounding forest within a given radius, 3) \textbf{forest connectivity}: the percentage of patches adjacent to each other. 
These factors will be assessed using the Anomaly Vegetation Change Detection (AVOCADO) algorithm, which is based on the R package "npphen" and with the R package "landscapemetrics" \citep{chavez2017npphen, hesselbarth2019landscapemetrics}. 

AVOCADO is designed to detect forest disturbance and regrowth based on satellite imagery in a semi-automated and continuous manner. It utilizes all available data without requiring outlier removal \citep{decuyperContinuousMonitoringForest2022}. As a time series algorithm, AVOCADO analyzes more than two satellite images to detect change. Compared to bi-temporal difference or supervised image classification methods, time series approaches offer greater precision in detecting small-scale forest changes by capturing the dynamic behavior of vegetation over time. This advantage is essential when analysing forest change dynamics within a smallholder agricultural landscape which is often characterized by complex vegetation dynamics caused by forest degradation and shifting cultivation. Unlike many other traditional time series algorithms, AVOCADO holds an additional benefit in being able to detect more than one disturbance or regrowth. Moreover, time series methods such as temporal segmentation and trend analysis resolve this issue by applying parametric functions which rely on the assumption of a regular phenological cycles. Instead of parametric function, AVOCADO uses the observed frequency values which enables for a greater flexibility to adapt to site specific conditions and accounts for natural phenological variability. 

% here a sentence on why is AVOCADO in a summary a good algorithm to determine forest connectivity, cover and age

The AVOCADO algorithm uses a reference vegetation from a nearby pixel known to have remained undisturbed throughout the time series. This reference pixel will be selected with the help of an expert with a long-term field experience in the area. 

The input to the AVOCADO algorithn will be Landsat derived Normalized Difference Moisture Index (NDMI) image. Landsat is selected due to its long historical archive. While Landsat spatial resolution (30m) is sometimes considered a hindrance compared to its temporal resolution, this is not a significant limitation for this study, as the resolution is comparable to the spatial scale of the investigated forest plots (25m). NDMI, derived from Landsat's near-infrared and short-wave infrared bands, will be used as it is able to capture gradual changes and regrowth in contrast to NDVI . The landsat images will be downloaded and pre-processd via the Google Earth Engine (GEE) plantform \citep{gorelickGoogleEarthEngine2017}. The pre-processing steps will involve the exclusion of images with cloud cover higher than 70\% and clipping of the images by the area of interest. 

The output of the algorithm is a raster map with an indication of the year since the last disturbance. This map will be used for a further evaluation of the forest age, forest cover, and forest connectivity using the landscapemetric or the lconnect package in R \citep{mestreLconnectPackageVersatile2023, hesselbarthLandscapemetricsOpensourceTool2019}. The satellite imagery used will be Landsat NDMI images.
\citep{gore}

The evaluation will be performed in R software \citep{R}.

\subsection{Statistical analysis}
To investigate how seed richness, evenness and seed dipersal mode are affected by forest age, cover and connectivity, a Linear Mixed Model will be performed. The model will be evaluated using $R^2$, RMSE and AIC.

 