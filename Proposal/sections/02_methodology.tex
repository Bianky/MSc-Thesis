
% METHODOLOGY
% •	Study area description, if any
% •	Justify certain choices of methods and datasets
% •	Be clear on what will be investigated using what metrics. For example, the research will study the uptake of solar panel installation for different provinces by comparing the number of households that have solar panel installations since 2015. 
% •	Be clear on how to validate the results and using what metrics. For example, assess the accuracy of detecting solar panels using overall accuracy, user’s accuracy, precision, etc.,
% •	Clearly describe any ground truth or reference data, e.g., how to access and how/what to collect
% •	page indication: 2 

\subsection{Study Area}
This research will be carried out in dry tropical forest located in Nizanda (16°39′ N, 95°00′ W), Mexican province of Oxaca and in the wet tropical forest located in Loma Bonita (16°01’ N, 90°55’), Mexican province of Chiapas. 

The dry tropical forest has a mean annual temperature of 27.7°C with a mean annual precipitation of 900mm. It is characterized as dry tropical deciduous forest \citep{hordijkLandUseHistory2024}. 

The wet forest has a mean annual temperature of 24°C and a mean annual precipitation of 3000mm. The vegetation varies between semideciduous and lowland tropical rain forest \citep{hordijkLandUseHistory2024}.

There is 20 (25m x 25m) plots established in both the dry and wet tropical forest. These plots were established on an recently abandoned pasture or agricultural land by the PANTROP project in March 2020 \citep{hordijkLandUseHistory2024}. If needed, plots were fenced off to prevent grazing. 



\subsection{Data Collection}

\subsubsection{Seeds}
To evaluate the seed diversity and dispersal mode in both forest sites (wet and dry), four litter traps of 50cm x 50cm were established in each plot. The litter traps were positioned 6m from the plot's edge in each corner of the plots. The litter is collected monthly since the establishment of the plots and the content of the four traps are combined into one sample per plot. 
Once it is collected, the litter traps are dried, weighted and sorted into leaves, twigs, reproductive tissues and others. Subsequently, seed counting and identification is performed with a seed catalogue provided by UNAM and with a support of an expert in the laboratory in Morelia. The seeds are classified into life form groups such as herbs, grasses and trees. Based on literature, the mode of dispersal is identified. 

For this research, a fieldwork will be conducted in the dry tropical forest in Nizanda, Mexico. 
Unfortunately, it is not possible to conduct fieldwork in the wet tropical forest in Loma Bonita due to the political situation at place. However, a data on wet tropical forest will be retrieved from the PANTROP project database to be able to analyse and compare wet and dry tropical forest.

\subsubsection{Forest Landscape Factors}
Landsat satellite imagery will be used for data acquiry on the forest landscape factors. Namely, forest age, forest cover and forest connectivity. Additionally, the forest landscape factors will be calculated using a time-series algorithm called AVOCADO and R packages landscapemetric and lconnect.

\subsection{Analysis}
To investigate the relationship Linear Mixed Model will be performed on dependent variable: seed richness, seed evenness and seed dispersal mode and the independent variables: forest age, cover and forest connectivity. The model will be evaluated using R2, RMSE and bias.

%In general, such new methods are promising in
%smallholder agricultural landscapes, that are %characterized by complex
%forest dynamics caused by shifting cultivation, forest degradation and
%large variety of agroecosystems

%Normalised difference moisture index (NDMI) at 40m pixel resolution
 