
% METHODOLOGY
% •	Study area description, if any
% •	Justify certain choices of methods and datasets
% •	Be clear on what will be investigated using what metrics. For example, the research will study the uptake of solar panel installation for different provinces by comparing the number of households that have solar panel installations since 2015. 
% •	Be clear on how to validate the results and using what metrics. For example, assess the accuracy of detecting solar panels using overall accuracy, user’s accuracy, precision, etc.,
% •	Clearly describe any ground truth or reference data, e.g., how to access and how/what to collect
% •	page indication: 2 

\subsection{Location}
This research will be carried out in dry tropical forest located in Nizanda (16°39′ N, 95°00′ W), Mexican province of Oxaca and in the wet tropical forest located in Loma Bonita (16°01’ N, 90°55’), Mexican province of Chiapas. 

The dry tropical forest has a mean annual temperature of 27.7°C with a mean annual precipitation of 900mm. It is characterized as dry tropical deciduous forest \citep{hordijkLandUseHistory2024}. 

The wet forest has a mean annual temperature of 24°C and a mean annual precipitation of 3000mm. The vegetation varies between semideciduous and lowland tropical rain forest \citep{hordijkLandUseHistory2024}.

There is 13 (25m x 25m) plots established in the dry tropical forest and 20 (25m x 25m) in the wet tropical forest. These plots were established on an recently abandoned pasture or agricultural land by the PANTROP project in 2020 \citep{hordijkLandUseHistory2024}.

This research will be composed of a fieldwork and a spatial analysis. The fieldwork will serve for data collection on the seed variables, and the spatial analysis will be a mean to acquire insights into the forest landscape variables. 
Additionally, the PANTROP database will be used to access historical data from previous years. 

\subsection{Fieldwork}
For this research, a fieldwork will be conducted in the dry tropical forest in Nizanda, Mexico. During the fieldwork, litter traps containing plant litter and seeds will be collected. This litter will be dried, weighted and sorted into leaves, twigs, reproductive tissue and others. 
Subsequently, seed counting and identification will be performed with a seed catalogue provided by UNAM in the laboratory in Morelia. The seeds will be classified into life form groups such as herbs, grasses and trees. Subsequentially, absed on literature, their mode of dispersal will be identified.

Unfortunately, it is not possible to conduct fieldwork in the wet tropical forest in Loma Bonita due to the political situation at place. However, a data on wet tropical forest will be retrieved from the PANTROP project database to be able to analyse and compare wet and dry tropical forest.

\subsection{Spatial analysis}
Landsat satellite imagery will be used for data acquiry on the forest landscape factors. Namely, forest age, forest cover and forest connectivity. Additionally, the forest landscape factors will be calculated using a time-series algorithm called AVOCADO. 

In general, such new methods are promising in
smallholder agricultural landscapes, that are characterized by complex
forest dynamics caused by shifting cultivation, forest degradation and
large variety of agroecosystems

Normalised difference moisture index (NDMI) at 40m pixel resolution
 