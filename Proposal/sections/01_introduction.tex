

% INTRODUCTION
% •	Start at a broader societal or scientific context (e.g., climate change, sustainability challenges) and then narrow down to the main topic of the thesis. 
% •	Include descriptions of:
    % •	what are general ways (or opportunities) to tackle certain issues (e.g., need accurate information to respond/mitigate climate change)
    % •	What kind of scientific data and methods can be used? 
    % •	What further possibilities are there? (e.g., new satellite data, new technical advancements)
    % •	Avoid including the thesis research goals here (there is a separate section – section 3)
% •	At least 5 peer-reviewed references ( preferably not older than 10 years)
% •	Page indication:1

% RESEARCH NEEDS
% • This section is generally more specific compared to the previous section focusing on questions like what needs to be researched and what the research gaps are. For example, there are new opportunities (e.g. new tools or data sources), but further study is needed to evaluate their use/value.  
% • This section is not to confuse with “societal needs”, it only focuses on research needs. The societal needs/context is already tackled in the introduction section. 
% • The research needs descriptions are often linked with the research questions of the proposal. In other words, with this thesis research, by answering the research questions, your work will contribute to bridging the current research gap. 
% • This section can be combined with the introduction section, but the main content needs to be present. 
% • At least 10 peer-reviewed references (preferably not older than 10 years) 
% • page indication: 1

% RESEARCH AIM AND RESEARCH QUESTIONS
% •	Apply the SMART principle for the research objective (Specific, Measurable, Attainable, Relevant and Time-bound)
% •	Research questions should be linked to research needs/gaps
% •	Avoid closed questions with “yes/no” answer and questions such as “how to” or “how can”. The latter questions involve certain choices in methods. Hence the obtained results can be only specific to the chosen steps/methods. 
% •	To define a clear research question, it is useful to imagine the expected answer to the questions and also the form of the answer: e.g., map, lists, tables, and figures. 
% •	See more explanations on research questions here (Brightspace discovery course for scientific reading and writing)


Tropical forests are the world's key ecosystems providing habitat for a wide  diversity of species, supporting local livelihoods and playing an integral role in the global ecological cycles such as carbon sequestration, water regulation, and nutrient cycling \citep{bormaCarbonContributionsSouth2022}. They host approximately two thirds of the world's flora and fauna, making them an essential habitat of the global biodiversity \citep{mulatuBiodiversityMonitoringChanging2017}. Tropical ecosystems are not only crucial for wildlife, but also for local livelihoods of millions of people who rely on their resources for food, medicine, and income \citep{bormaCarbonContributionsSouth2022}. Despite their importance, over half of the tropical forests have been deforested and converted into other land use types, with agriculture being the most common, leading to losses of these vital ecosystem services \citep{chazdonNaturalRegenerationTool2016}. Given the growing pressure on tropical forests, natural forest regeneration presents a promising solution for ecological restoration \citep{hordijkLandUseHistory2024}. 
%is an effective tool for various ecological restoration efforts such as climate change mitigation, biodiversity conservation, or soil health restoration \citep{hordijkLandUseHistory2024}. % here we could talk more specifically about ecological efforts and societal 
%It supports global ambitions such as the restoration of 3.5 million $km^2$ of degraded land \citep{holl2017restoring} and aligns with the UN Decade on Ecosystem Restoration, which aims to reach land degradation neutrality by 2030 \citep{waltham2020decade}. 
Natural forest regeneration is a spontaneous recovery process during which native tree species colonize land following a human-induced or natural disturbances \citep{crouzeillesEcologicalRestorationSuccess2017}. This process is particularly significant in tropical regions, where intensive agriculture practices often lead to resource depletion and thus land abandonment. Over time, vegetation and soil recover resulting in a formation of a secondary forest \citep{chazdonSecondGrowthPromise2014}. However, the success of natural regeneration is strongly influenced by the spatial context such as the degree of landscape fragmentation and the proximity to surrounding forest \citep{arroyo-rodriguezMultipleSuccessionalPathways2017}. This study aims to investigate the impact of structural characteristics of the forest landscape on a fundamental component of natural forest regeneration: seeds.

%
%As described by \citet{crouzeillesEcologicalRestorationSuccess2017}, natural forest regeneration is a spontaneous recovery process during which native tree species colonize land after human-induced or natural disturbance.
  % potentially, elaboration on how and why does this happen?

%Subsequently to land abandonment, the soil and vegetation naturally recover resulting in a secondary forest \citep{hordijkLandUseHistory2024}. % here we could add more from Chazdon, 2016 on the importance of tropics in the world
%Moreover, a higher degree of fragmentation limits the movement possibility of animal despersers and lowerproximity to a surrounding forest 

% on secondary succession
In ecological terms natural regeneration is known as secondary succession - a directional change in ecosystem attributes over time following a disturbance. Unlike primary succession, secondary succession benefits from land legacies such as well-established soils and the presence of reproductive units, propagules \citep{poorterSuccessionalTheories2023}. % here a potential to elaborate on the history of succession theories development
Successional theory has evolved over time with 
key frameworks proposed by \citet{pickettHierarchicalConsiderationCauses1987} and \citet{poorterComprehensiveFrameworkVegetation2024}, identifying three main successional mechanisms: (1) site availability, (2) species availability, and (3) species performance. The site availability depends on the type, intensity and duration of a disturbance which in turn impacts the seed bank and the remnant vegetation \citep{poorterComprehensiveFrameworkVegetation2024}. Species availability is regulated by propagules that remain at the site after the disturbance and those dispersed into the site \citep{gleasonIndividualisticConceptPlant1926, dentUnitingNicheDifferentiation2021}. Finally, species performance is determined by species traits, biotic interactions, and the abiotic environment \citep{poorterComprehensiveFrameworkVegetation2024}. 

Among these, species availability often has the strongest influence on the speed and the direction of succession \citep{poorterSuccessionalTheories2023, dentUnitingNicheDifferentiation2021}. However, due to common agricultural land history, remnant propagules such as seedbanks and resprouts might be limited or depleted. Thus, leaving the species availability mechanism dependent on propagules dispersed into the site, also known as seed rain. The diversity and abundance of seeds in the seed rain are thereafter important factors for secondary succession. They are determined by biotic dispersers such as birds and mammals and abiotic dispersers such as wind and water \citep{poorterComprehensiveFrameworkVegetation2024}. The mode of dispersal can further influence where and how far are seeds deposited. Early successional stages are most commonly characterized by wind, bird and bats dispersal, whereas later successional stages, when seed abundance increases are more commonly characterized by larger mammals dispersal \citep{dentUnitingNicheDifferentiation2021}. Moreover, dispersal mode can also vary between wet and dry tropical forest. In dry tropical forest, seeds are more likely to be smaller and dispersed by wind, whereas in wet tropical forest, seeds tend to be covered in fleshy fruit and dispersed by frugivores . The mode of seed dispersal and therefore the abundance and diversity of seeds is, however, in turn dependent on the landscape characteristics of the forest. 

%Wind dispersed species are especially common in degraded landscapes, however animal dispersed species are necessary for driving a diverse succession 

The shifting land use change in the tropical region often results in a loss of forest cover and land fragmentation which reduces forest connectivity and limits the opportunities for movement. Consequently, this lack of connectivity can significantly hinder seed dispersal and establishment \citep{arroyo-rodriguezMultipleSuccessionalPathways2017}.

O%ne of the most important propagules by which species availability is regulated are seeds. Seeds are a fundamental component of secondary succession. As propagules, they function as the source for the development of new plants \citep{poorterSuccessionalTheories2023}. In the first stages of secondary succession, most of the seedlings are sourced from the surrounding forest, but also from the soil seed bank and resprouts of the remnant vegetation \citep{huancanunezSeedrainsuccessionalFeedbacksWet2021}.

The forest landscape factors such as the age of the forest, the forest cover and the forest connectivity play a critical role in the seed diversity and distribution. If the occurrence of surrounding forest is scarce, the chance of seed dispersal and establishment, and therefore succession, is hindered \citep{hordijkLandUseHistory2024}. Additionally, the forest connectivity is particularly important in this stage, when seeds are mainly sourced from the surrounding forest. In poorly connected forests, animal dispersers have limited options for movement and therefore, transportation of seeds.  Moreover, the distance to nearby seed source and the presence of natural corridors can impact the composition and abundance of seeds available for regeneration significantly \citep{dentUnitingNicheDifferentiation2021, arroyo-rodriguezMultipleSuccessionalPathways2017}.

%While these mechanisms are broadly applicable, the factors shaping the successional pathway might vary substantially between ecosystems. For instance, in tropical forests, the process differs between dry and wet tropical forest. In wet tropical forest, succession is more dependent on the landscape connectivity and the availability of seed dispersers, as most tree species rely on animals for dispersal. In contrast, tropical dry forest is dominated by wind-dispersed species and trees that resprout from existing root system \citep{hordijkLandUseHistory2024, lohbeckSuccessionalChangesFunctional2013}.

% a better link would be nice, between these two paragraphs

%In this MSc research I wish to focus on the second successional mechanism, the species %availability. Species availability has always been one of the integral components of %successional theory since the beginning of last century \citep{gleasonIndividualisticConceptPlant1926, arroyo-rodriguezMultipleSuccessionalPathways2017, dentUnitingNicheDifferentiation2021}. 

%The size of the
%largest forest patch and connectivity had the most effect on change
%in tree community attributes during 3 years of succession (Table 3),
%indicating that not only forest presence and quantity but also forest
%configuration in the landscape affect tree establishment. - Hordijk, Land use

% and then how it is important for society and the aim of the study
%Understanding the mechanisms of secondary succession broadens our knowledge on how ecosystems respond to a disturbance. This is particularly important in the age of Anthropocene during which global land use and climate change cause severe disruptions in the ecosystems. By understanding the dynamics present in secondary succession, we can design and implement more effective ecosystem restoration strategies \citep{poorterSuccessionalTheories2023}. Therefore, the aim of this study is to understand how forest landscape factors affect seed diversity and dispersal in a dry and wet tropical forest during early succession.

%"Community structure: species richness increases due to the arrival of new species, species evenness increases because dominant pioneer species decline in abundance, and there is an increase in vertical stratification and spatial heterogeneity. The carbon and nitrogen pool size increase due to autotrophic assimilation, nutrient uptake, and biological nitrogen fixation, and an increasing part of the" - successional theories

% definition and further on forest surroundings

%As the successional age increases, the number of immigrant seeds decreases and the seed rain composition of present trees starts to resemble those of adult trees \citep{chazdonNaturalRegenerationTool2016}. However, the forest cover and connectivity remain vital for the seed richness and abundance. Moreo

%However, if trees occurrence is scarce in the surrounding area, the succession is hindered \citep{hordijk2023efectos}. With increasing tree abundance and more forested surroundings, the speed and potential for recovery also rise \citep{arroyo-rodriguezMultipleSuccessionalPathways2017}. The surrounding forest and connectivity are particularly important during the first stage of secondary succession when seeds are mainly sourced from the nearby vegetation by biotic or abiotic factors. However, with increasing successional age, the proportion of seeds sourced from surrounding forest decreases \citep{huancanunezSeedrainsuccessionalFeedbacksWet2021}. 

%Unlike seeds from a temperate forest, tropical forest seeds are animal dispersed \citet{dentUnitingNicheDifferentiation2021}. However, the main mechanism of dispersal changes through the different succession stages. During the early successional stage, wind plays an important role in seed dispersal

%Landscapes
%at one extreme: landscape dominated by forested matrices are mainly determined %by local variation of environmental factors since seed abundance is relatively %high, whereas landscape dominated by various land use types will be mainly %determined by factors operating at landscape and regional scale, especially %those which affect production, dispersal, and predating of propagules


% something that might be related to the age of the forest -> "This confirms that early-successional species invest in many small seeds that can travel large distances (e.g., by wind), whereas late-successional species invest in fruits that attract biotic dispersers to allow directional dispersal" (Lohbeck)

% gap
Altough a substantial research has been published on the mechanisms of secondary succession and to its related landscape factors, few studies have investigated the effects of their impact on seed diversity and dispersal comparing dry and wet tropical forest ecosystems. To address this gap, the following research question has been identified: \textit{How does forest age, cover and connectivity affect seed abundance, richness and dispersal mode in dry and wet secondary tropical forests during early succession?}

%The terms investigated in this research are defined as following:\\\\
%\textit{Forest landscape factor terms.}\\
%\textbf{Forest age:} the number of years since a forest's establishment.\\
%\textbf{Forest cover:} spatial extent covered with trees.\\
%\textbf{Forest connectivity:} the degree of agglomerations between forest patches.\\
%\\
%\textit{Seed diversity and dispersal terms.} \\
%\textbf{Seed richness:} measure of the number of different seed species.\\
%\textbf{Seed evenness:} measure of the relative abundance of different seed species.\\
%\textbf{Seed dispersal mode:} the mechanisms by which a seed is moved from a source to an establishment site.\\

It is hypothesized that seed richness and evenness will increase with forest age in both dry and wet tropical forests, as early successional species are gradually replaced by later successional species over time \citep{chazdonNaturalRegenerationTool2016}. It is further hypothesized that forest connectivity and cover will have a negative impact on seed biotic dispersal in dry tropical forests due to the higher occurrence of abiotic dispersal in these areas. In contrast, in wet tropical forests, forest connectivity and cover are expected to positively impact biotic dispersal, as biotic dispersal mechanisms are more dominant. Additionally, it is expected that forest connectivity and cover will have a positive effect on seed richness and diversity in wet tropical forests, as biotic dispersal increases the potential for seed movement and establishment \citep{dentUnitingNicheDifferentiation2021}.
