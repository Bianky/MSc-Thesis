

% INTRODUCTION
% •	Start at a broader societal or scientific context (e.g., climate change, sustainability challenges) and then narrow down to the main topic of the thesis. 
% •	Include descriptions of:
    % •	what are general ways (or opportunities) to tackle certain issues (e.g., need accurate information to respond/mitigate climate change)
    % •	What kind of scientific data and methods can be used? 
    % •	What further possibilities are there? (e.g., new satellite data, new technical advancements)
    % •	Avoid including the thesis research goals here (there is a separate section – section 3)
% •	At least 5 peer-reviewed references ( preferably not older than 10 years)
% •	Page indication:1

% RESEARCH NEEDS
% • This section is generally more specific compared to the previous section focusing on questions like what needs to be researched and what the research gaps are. For example, there are new opportunities (e.g. new tools or data sources), but further study is needed to evaluate their use/value.  
% • This section is not to confuse with “societal needs”, it only focuses on research needs. The societal needs/context is already tackled in the introduction section. 
% • The research needs descriptions are often linked with the research questions of the proposal. In other words, with this thesis research, by answering the research questions, your work will contribute to bridging the current research gap. 
% • This section can be combined with the introduction section, but the main content needs to be present. 
% • At least 10 peer-reviewed references (preferably not older than 10 years) 
% • page indication: 1

% RESEARCH AIM AND RESEARCH QUESTIONS
% •	Apply the SMART principle for the research objective (Specific, Measurable, Attainable, Relevant and Time-bound)
% •	Research questions should be linked to research needs/gaps
% •	Avoid closed questions with “yes/no” answer and questions such as “how to” or “how can”. The latter questions involve certain choices in methods. Hence the obtained results can be only specific to the chosen steps/methods. 
% •	To define a clear research question, it is useful to imagine the expected answer to the questions and also the form of the answer: e.g., map, lists, tables, and figures. 
% •	See more explanations on research questions here (Brightspace discovery course for scientific reading and writing)

Natural forest regeneration is an effective tool for various ecological restoration efforts such as climate change mitigation, biodiversity conservation, or soil health restoration \citep{hordijkLandUseHistory2024}. % here we could talk more specifically about ecological efforts and societal 
As described by \citet{crouzeillesEcologicalRestorationSuccess2017}, natural forest regeneration is a spontaneous recovery process during which native tree species colonize land after human-induced or natural disturbance.
The significance of natural regeneration is particularly important in the tropical region, where shifting land use change often results in land abandonment \citep{chazdonSecondGrowthPromise2014}. % potentially, elaboration on how and why does this happen?
Subsequently to land abandonment, the soil and vegetation naturally recover resulting in a secondary forest \citep{hordijkLandUseHistory2024}. % here we could add more from Chazdon, 2016 on the importance of tropics in the world

% on secondary succession
In ecological terms natural regeneration is a process called secondary succession. Succession is defined as a directional change over time presented by one or multiple ecosystem attributes consecutive to perturbation. Additionally, unlike primary succession, secondary succession is characterized by various land legacies such as well-established soils and the presence of reproductive units, propagules \citep{poorterSuccessionalTheories2023}. % here a potential to elaborate on the history of succession theories development
Based on \citet{pickettHierarchicalConsiderationCauses1987} and \citet{poorterComprehensiveFrameworkVegetation2024} theoretical frameworks, succession is determined by three main successional mechanisms: (1) site availability, (2) species availability, and (3) species performance. The site availability is determined by the intensity and duration of a disturbance. Secondly, the species availability is regulated by propagules that remained at the site after the disturbance and by propagules that were dispersed to the patch \citet{gleasonIndividualisticConceptPlant1926, dentUnitingNicheDifferentiation2021}. Lastly, species performance is determined by species traits, biotic interactions, and the abiotic environment \cite{poorterComprehensiveFrameworkVegetation2024}. Typically, species availability has a stronger influence on the speed and direction of succession than the environmental requirements of plant species. This is particularly the case for fragmented landscapes in which seeds and seed dispersers can occur infrequently or they can be completely absent \citep{poorterSuccessionalTheories2023, dentUnitingNicheDifferentiation2021}. 

Although successional mechanisms which compose successional theories are generally applicable on a global scale, the factors that determine the successional pathway might vary substantially. The process of succession in tropical dry forest differs from the one in tropical wet forest. In wet tropical forest, secondary succession is more impacted by the landscape context and the presence of seed disperers. This is due to the fact that majority of tree species in wet tropical forest are animal dispersed. In dry forest, on the other hand, seed dispersal is mainly driven by wind and trees' resprouts \citep{hordijkLandUseHistory2024, lohbeckSuccessionalChangesFunctional2013}. Additionally, the speed of succession is higher in wet tropical forest compared to dry tropical forest, which might be due to the year-round growing season \citep{hordijk2023efectos}.

%In this MSc research I wish to focus on the second successional mechanism, the species %availability. Species availability has always been one of the integral components of %successional theory since the beginning of last century \citep{gleasonIndividualisticConceptPlant1926, arroyo-rodriguezMultipleSuccessionalPathways2017, dentUnitingNicheDifferentiation2021}. 


% and then how it is important for society and the aim of the study
Understanding the mechanisms of secondary succession broadens our knowledge on how ecosystems responds to a disturbance. This is particularly important in the age of Anthropocene during which global land use and climate change cause severe disruptions in the ecosystems. By understanding the dynamics present in secondary succession, we can design and implement more effective ecosystem restoration strategies \cite{poorterSuccessionalTheories2023}. Therefore, the aim of this study is to understand how forest landscape factors affect seed diversity and dispersal in a dry and wet tropical forest during the early succession. 

% definition and further on forest surroundings
Seeds play a fundamental role in the process of secondary succession. They are a type of a propagule which function as a source for a development of a new plant \citep{poorterSuccessionalTheories2023}. However, if there is limited occurrence of trees in the surrounding area, the succession is restricted \citep{hordijk2023efectos}. Therefore, with increasing tree abundance and forested surroundings, the speed and potential of recovery also increases \citet{arroyo-rodriguezMultipleSuccessionalPathways2017}. The surrounding forest and connectivity are particularly important during the first stage of secondary succession when seeds are mainly sourced from the nearby vegetation. However, with increasing successional age, the proportion of seeds sourced from surrounding forest decreases 

The forest connectivity is particularly important for wet tropical forests in which seed dispersal factors are mainly biotic such as fugivores, a fruit eating herbivores. Additionally, the surrounding forest might also function as a source of habitat for seed dispersers and pollinators \citep{hordijk2023efectos, arroyo-rodriguezMultipleSuccessionalPathways2017}. In dry tropical forests, seeds are usually dispersed by abiotic factors such as wind. 

% something that might be related to the age of the forest -> "This confirms that early-successional species invest in many small seeds that can travel large distances (e.g., by wind), whereas late-successional species invest in fruits that attract biotic dispersers to allow directional dispersal" (Lohbeck)

% gap
Altough, there has been a substantial amount of research published on the mechanisms of secondary succession and to its related landscape factors, a research which investigates the effect of the forest landscape factors on seed diversity and dispersal comparing the wet and dry tropical forest is still missing. 
% research question and hypothesis
Therefore, the following research question has been identified: How does forest age, cover and connectivity affect seed abundance, richness and dispersal mechanism in dry and wet secondary tropical forests during early succession? 
Seed richness and evenness will increase with with forest age \citep{chazdonSecondGrowthPromise2014}
