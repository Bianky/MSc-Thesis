

% INTRODUCTION
% •	Start at a broader societal or scientific context (e.g., climate change, sustainability challenges) and then narrow down to the main topic of the thesis. 
% •	Include descriptions of:
    % •	what are general ways (or opportunities) to tackle certain issues (e.g., need accurate information to respond/mitigate climate change)
    % •	What kind of scientific data and methods can be used? 
    % •	What further possibilities are there? (e.g., new satellite data, new technical advancements)
    % •	Avoid including the thesis research goals here (there is a separate section – section 3)
% •	At least 5 peer-reviewed references ( preferably not older than 10 years)
% •	Page indication:1

% RESEARCH NEEDS
% • This section is generally more specific compared to the previous section focusing on questions like what needs to be researched and what the research gaps are. For example, there are new opportunities (e.g. new tools or data sources), but further study is needed to evaluate their use/value.  
% • This section is not to confuse with “societal needs”, it only focuses on research needs. The societal needs/context is already tackled in the introduction section. 
% • The research needs descriptions are often linked with the research questions of the proposal. In other words, with this thesis research, by answering the research questions, your work will contribute to bridging the current research gap. 
% • This section can be combined with the introduction section, but the main content needs to be present. 
% • At least 10 peer-reviewed references (preferably not older than 10 years) 
% • page indication: 1

% RESEARCH AIM AND RESEARCH QUESTIONS
% •	Apply the SMART principle for the research objective (Specific, Measurable, Attainable, Relevant and Time-bound)
% •	Research questions should be linked to research needs/gaps
% •	Avoid closed questions with “yes/no” answer and questions such as “how to” or “how can”. The latter questions involve certain choices in methods. Hence the obtained results can be only specific to the chosen steps/methods. 
% •	To define a clear research question, it is useful to imagine the expected answer to the questions and also the form of the answer: e.g., map, lists, tables, and figures. 
% •	See more explanations on research questions here (Brightspace discovery course for scientific reading and writing)

Natural forest regeneration is an effective tool for various ecological restoration efforts such as climate change mitigation, biodiversity conservation, or soil health restoration \citep{hordijkLandUseHistory2024}. % here we could talk more specifically about ecological efforts and societal 
As described by \citet{crouzeillesEcologicalRestorationSuccess2017}, natural forest regeneration is a spontaneous recovery process during which native tree species colonize land after human-induced or natural disturbance.
The significance of natural regeneration is particularly important in the tropical region, where shifting land use change often results in land abandonment \citep{chazdonSecondGrowthPromise2014}. % potentially, elaboration on how and why does this happen?
Subsequently to land abandonment, the soil and vegetation naturally recover resulting in a secondary forest \citep{hordijkLandUseHistory2024}. % here we could add more from Chazdon, 2016 on the importance of tropics in the world

% on secondary succession
In ecological terms natural regeneration is a process called secondary succession. Succession is defined as a directional change over time presented by one or multiple ecosystem attributes consecutive to perturbation. Additionally, unlike primary succession, secondary succession is characterized by various land legacies such as well-established soils and the presence of reproductive units, propagules \citep{poorterSuccessionalTheories2023}. % here a potential to elaborate on the history of succession theories development
Based on \citet{pickettHierarchicalConsiderationCauses1987} and \citet{poorterComprehensiveFrameworkVegetation2024} theoretical frameworks, succession is determined by three main successional mechanisms: (1) site availability, (2) species availability, and (3) species performance. In the early stage of succession, these three successional mechanisms operate consecutively. In this MSc research I wish to focus on the second successional mechanism, the species availability. Species availability has always been one of the integral components of successional theory since the beginning of last century \citep{}


% roughly on forest surroundings and seeds
and forest surrounding is very important and seeds

% and then how it is important for society
Understanding secondary succession broadens our knowledge on how ecosystems responds to a disturbance. This is particularly important in the age of Anthropocene during which global land use and climate change cause severe disruptions in the ecosystems. By understanding the dynamics present in secondary succession, we can design and implement more effective ecosystem restoration strategies \cite{poorterSuccessionalTheories2023}.

% the aim of the study
Therefore, the aim of this study is to understand how forest landscape dynamics affect seed diversity in a dry and wet tropical forest during the early succession. 

% definition and further on forest surroundings
When talking about site availability, the first out of the three main successional mechanisms, this paper aims to focus on the following factors: forest age, cover and connectivity \citep{poorterComprehensiveFrameworkVegetation2024}.
Forest landscape dynamics (forest cover, age, connectivity)

% definition and further on seeds
Seed diversity (richness, evenness, dispersal)

% further on wet and dry

% gap

% research question and hypothesis
How does forest age, cover and connectivity affect seed abundance, richness and dispersal mechanism dry and wet secondary tropical forests during early succession?