

% INTRODUCTION
% •	Start at a broader societal or scientific context (e.g., climate change, sustainability challenges) and then narrow down to the main topic of the thesis. 
% •	Include descriptions of:
    % •	what are general ways (or opportunities) to tackle certain issues (e.g., need accurate information to respond/mitigate climate change)
    % •	What kind of scientific data and methods can be used? 
    % •	What further possibilities are there? (e.g., new satellite data, new technical advancements)
    % •	Avoid including the thesis research goals here (there is a separate section – section 3)
% •	At least 5 peer-reviewed references ( preferably not older than 10 years)
% •	Page indication:1

The alarming rate of tropical forest degradation and deforestation in human modified landscapes
have led to a massive loss of primary forests. Forest conversion to other types of land-uses results in
the loss of biodiversity, land degradation and contributes to global warming. However, forests can
regrow as secondary forests, which have increased in area in tropical regions as a result of land
abandonment. These naturally regenerated forests recover rapidly, attaining 78\% of the values of
neighbouring old-growth forests after only 20 years


Tropical forests are world's key ecosystems accommodating a wide diversity of species, ..., and playing a fundamental role in the global carbon and water cycles \citep{bormaCarbonContributionsSouth2022}. However, over half of the tropical forests have been deforested and converted into other land use types leading to losses of these vital ecosystem services \citep{chazdonNaturalRegenerationTool2016}. 

Therefore, natural forest regeneration is an essential tool in efforts to restore soil and ecosystem health, and to mitigate climate change \citep{hordijkLandUseHistory2024}.  resto In the tropics over a half of the forest cover has been converted to other land use type \citep{chazdonNaturalRegenerationTool2016}.

during natural regeneration forest regrow into secondary forests

According to ... over half Why is it essentail According to \citet{chazdonNaturalRegenerationTool2016}, over half of the tropical forests have been converted to other land uses the cover of regeenrating forests
According to \citet{crouzeillesEcologicalRestorationSuccess2017}, natural forest regeneration is defined as "spontaneous recovery of native tree species that colonize and establish in abandoned field or natural disturbances".

But not only for climate, but also for biodiversity conservation, soil health restoration and ecosystem services

Natural forest restoration is an essential tool in efforts against climate change.

Tropical forests, world's vital ecosystems, are being degraded and deforested at an alarming rate leading to a vast loss of primary forests \citep{lopez-bedoyaPrimaryForestLoss2022}. The conversion of forest to other land use types leads to biodiversity loss, soil and ecosystem degradation and land use emissions. however

Tropical forests, world's vital ecosystems, are being degraded and deforested at an alarming rate leading to a loss of biodiversity, soil erosion, and land use emissions \citep{lopez-bedoyaPrimaryForestLoss2022}. However, they can regrow into a secondary forest during the process of secondary succession. 

natural forest regeneration 

Its quiet common for the tropics as the agricultural land after being used gets abandoned

- tropical fores loss and recovery
- secondary forests
- natural regeneration : succession

- forest dynamics
- seeds

Natural regeneration is an effective tool for various ecological restoration efforts such as climate change mitigation, biodiversity conservation, or soil health restoration \citep{hordijkLandUseHistory2024}. The importance of natural regeneration is particularly important in the tropical region, where shifting land use change often results in land abandonment . A

it is important to study the ecoogical dynamics of recovering forest as it is an essential component of ecosystem 

% RESEARCH NEEDS
% • This section is generally more specific compared to the previous section focusing on questions like what needs to be researched and what the research gaps are. For example, there are new opportunities (e.g. new tools or data sources), but further study is needed to evaluate their use/value.  
% • This section is not to confuse with “societal needs”, it only focuses on research needs. The societal needs/context is already tackled in the introduction section. 
% • The research needs descriptions are often linked with the research questions of the proposal. In other words, with this thesis research, by answering the research questions, your work will contribute to bridging the current research gap. 
% • This section can be combined with the introduction section, but the main content needs to be present. 
% • At least 10 peer-reviewed references (preferably not older than 10 years) 
% • page indication: 1

The research, the aim of this study

% RESEARCH AIM AND RESEARCH QUESTIONS
% •	Apply the SMART principle for the research objective (Specific, Measurable, Attainable, Relevant and Time-bound)
% •	Research questions should be linked to research needs/gaps
% •	Avoid closed questions with “yes/no” answer and questions such as “how to” or “how can”. The latter questions involve certain choices in methods. Hence the obtained results can be only specific to the chosen steps/methods. 
% •	To define a clear research question, it is useful to imagine the expected answer to the questions and also the form of the answer: e.g., map, lists, tables, and figures. 
% •	See more explanations on research questions here (Brightspace discovery course for scientific reading and writing)

\subsection{Research questions}

How does forest age, cover and connectivity affect seed abundance, richness and dispersal mechanism in early dry and wet secondary forests?