
% TIME SCHEDULE
% •	Plan the time schedule considering the main methods or research questions
% •	Indicate if working part-time or other engagements such as a teaching assistant job, holidays, etc.
% •	Consider also the main milestones such as the proposal, midterm presentation, research questions, validation and writing /finalising
% •	Preferable to use a Gantt Chart, project planning timeline, network diagram, Kanban board or any other visual aid for project management
% •	page indication: 0.5 

This section elaborates on the proposed time frame for this MSc thesis research. The start date of this research is 01.02.2025 and the predicted end date is 30.09.2025. For a detailed overview, see Table \ref{tab:timeschedule}.

\begin{table}[h]
\captionof{table}\\{\textit{Proposed time schedule for the MSc Thesis Research}}\\

\begin{tabular}{lll}\hline
\multicolumn{1}{c}{\textbf{When}} & \multicolumn{1}{c}{\textbf{What}}   & \multicolumn{1}{c}{\textbf{Where}} \\\hline
01/02/2025 - 14/02/2025           & Proposal writing + presentation     & Wageningen (NL)                    \\
16/02/2025 - 02/03/2025           & Fieldwork in Nizanda                & Nizanda, Oxaca (MX)                \\
03/03/2025 - 20/04/2025           & Labwork in Morelia                  & Morelia, Michoacan (MX)            \\
%07/04/2025 - 20/04/2025           & Time off                 &           \\
21/04/2025 - 18/07/2025           & Data processing + analysis          & Wageningen (NL)                     \\
%19/07/2025 - 31/08/2025           & Time off                 &           \\

01/09/2025 - 30/09/2025           & Report writing + final presentation & Wageningen (NL)\\\hline                 
\end{tabular}\label{tab:timeschedule}
\end{table}

This is a proposed and expected time frame for the research. However, this research contains fieldwork, which might result in an occurrence of unforeseen circumstance due to which an adaptation of the schedule might be needed. 