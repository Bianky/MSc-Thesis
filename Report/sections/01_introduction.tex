

% INTRODUCTION
% •	Start at a broader societal or scientific context (e.g., climate change, sustainability challenges) and then narrow down to the main topic of the thesis. 
% •	Include descriptions of:
    % •	what are general ways (or opportunities) to tackle certain issues (e.g., need accurate information to respond/mitigate climate change)
    % •	What kind of scientific data and methods can be used? 
    % •	What further possibilities are there? (e.g., new satellite data, new technical advancements)
    % •	Avoid including the thesis research goals here (there is a separate section – section 3)
% •	At least 5 peer-reviewed references ( preferably not older than 10 years)
% •	Page indication:1

% RESEARCH NEEDS
% • This section is generally more specific compared to the previous section focusing on questions like what needs to be researched and what the research gaps are. For example, there are new opportunities (e.g. new tools or data sources), but further study is needed to evaluate their use/value.  
% • This section is not to confuse with “societal needs”, it only focuses on research needs. The societal needs/context is already tackled in the introduction section. 
% • The research needs descriptions are often linked with the research questions of the proposal. In other words, with this thesis research, by answering the research questions, your work will contribute to bridging the current research gap. 
% • This section can be combined with the introduction section, but the main content needs to be present. 
% • At least 10 peer-reviewed references (preferably not older than 10 years) 
% • page indication: 1

% RESEARCH AIM AND RESEARCH QUESTIONS
% •	Apply the SMART principle for the research objective (Specific, Measurable, Attainable, Relevant and Time-bound)
% •	Research questions should be linked to research needs/gaps
% •	Avoid closed questions with “yes/no” answer and questions such as “how to” or “how can”. The latter questions involve certain choices in methods. Hence the obtained results can be only specific to the chosen steps/methods. 
% •	To define a clear research question, it is useful to imagine the expected answer to the questions and also the form of the answer: e.g., map, lists, tables, and figures. 
% •	See more explanations on research questions here (Brightspace discovery course for scientific reading and writing)

% General
Tropical forests are the world's key ecosystems, providing habitat for a wide diversity of species, supporting local livelihoods, and playing an integral role in global ecological cycles such as carbon sequestration, water regulation, and nutrient cycling \citep{bormaCarbonContributionsSouth2022}. They host approximately two-thirds of the world's flora and fauna, making them essential habitats for global biodiversity \citep{mulatuBiodiversityMonitoringChanging2017}. Tropical ecosystems are not only crucial for wildlife, but also for local livelihoods of millions of people who rely on their resources for food, medicine, and income \citep{bormaCarbonContributionsSouth2022}. Despite their importance, over half of the tropical forests have been deforested and converted into other land use types, with agriculture being the most common, leading to loss of these vital ecosystem services \citep{chazdonNaturalRegenerationTool2016, arroyo-rodriguezMultipleSuccessionalPathways2017}. Given the growing pressure on tropical forests, natural forest regeneration presents a promising solution for ecological restoration \citep{hordijkLandUseHistory2024}. Natural forest regeneration is a spontaneous recovery process during which native tree species colonize land following human-induced or natural disturbances \citep{crouzeillesEcologicalRestorationSuccess2017}. This process is particularly significant in tropical regions, where intensive agriculture practices often lead to resource depletion and thus to land abandonment. Over time, vegetation and soil can recover, resulting in the formation of a secondary forest \citep{chazdonSecondGrowthPromise2014}. However, the success of natural regeneration is strongly influenced by tree seed inflow, which relies on the availability of tree seed sources and dispersers in the surrounding landscape. Tree seed inflow is therefore dependent on the spatial factors, such as the degree of landscape fragmentation and the proximity to surrounding forest \citep{arroyo-rodriguezMultipleSuccessionalPathways2017}. Although substantial research has been published on the importance of the surrounding forest for tree seed inflow during natural forest regeneration \citep{dentUnitingNicheDifferentiation2021, arroyo-rodriguezMultipleSuccessionalPathways2017, v.h.safarLandscapeOpennessHas2022}, there is an existing knowledge gap on how the surrounding forest impacts tree seed inflow in wet versus dry tropical forests. Therefore, this study aims to investigate the effects of forest landscape cover, connectivity, and age on tree seed abundance and diversity, and compare these effects between wet and dry tropical secondary forests.

% On secondary succession
In ecological terms, the transition from herbaceous to woody plants dominated vegetation is known as secondary succession - a directional change in ecosystem attributes over time following a disturbance. Unlike primary succession, secondary succession benefits from land legacies such as well-established soils and the presence of propagules, the reproductive units of plants \citep{poorterSuccessionalTheories2023}. % here a potential to elaborate on the history of succession theories development
Successional theory has evolved over time, with 
key frameworks proposed by \citet{pickettHierarchicalConsiderationCauses1987} and \citet{poorterComprehensiveFrameworkVegetation2024}, identifying three main successional mechanisms: (1) site availability, (2) species availability, and (3) species performance. Site availability depends on the type, intensity and duration of a disturbance, which in turn impacts seed bank and remnant vegetation \citep{poorterComprehensiveFrameworkVegetation2024}. Species availability is regulated by propagules that remain at the site after the disturbance and those dispersed into the site \citep{gleasonIndividualisticConceptPlant1926, dentUnitingNicheDifferentiation2021}. Finally, species performance is determined by species traits, biotic interactions, and the abiotic environment \citep{poorterComprehensiveFrameworkVegetation2024}. 

% On seeds
Among these, species availability often has the strongest influence on the speed and the direction of the first phase of succession \citep{poorterSuccessionalTheories2023, dentUnitingNicheDifferentiation2021}. However, due to the land history of common slash and burn agriculture in the tropics, remnant propagules such as seedbanks and resprouts might be limited or depleted \citep{bezerraDrasticImpoverishmentSoil2022}. Thus, leaving the species availability mechanism dependent on propagules dispersed into the site, also known as seed rain. The diversity and abundance of seeds in the seed rain are thereafter important factors for secondary succession. They are determined by biotic dispersers such as birds and mammals and abiotic dispersers such as wind and water \citep{poorterComprehensiveFrameworkVegetation2024}. However, the mode of seed dispersal and therefore the abundance and diversity of seeds is, in turn, dependent on the landscape surrounding the regenerating forest site. 

% The dominant mode of dispersal might vary during different stages of succession. Early successional stages are most commonly characterized by wind, birds and bats dispersal, whereas later successional stages, when seed abundance increases are more commonly characterized by larger mammals dispersal \citep{dentUnitingNicheDifferentiation2021}. Moreover, dispersal mode can also vary between wet and dry tropical forest. In dry tropical forest, seeds are more likely to be smaller and dispersed by wind, whereas in wet tropical forest, seeds tend to be covered in fleshy fruit and dispersed by frugivores \citep{vieiraPrinciplesNaturalRegeneration2006, dentUnitingNicheDifferentiation2021}. The mode of seed dispersal and therefore the abundance and diversity of seeds is, however, in turn dependent on the landscape surrounding the regenerating forest site. 

%Wind dispersed species are especially common in degraded landscapes, however animal dispersed species are necessary for driving a diverse succession 

% On landscape factors
% The landscape factors such as the age of the forest, the forest cover, and the forest connectivity play a critical role in the seed abundance, diversity and distribution. Moreover, the surrounding forest functions as a seed source and as a habitat for seed dispersers and pollinators \citep{hordijkLandUseHistory2024, arroyo-rodriguezMultipleSuccessionalPathways2017}. Older forests tend to have higher diversity and richness of seeds and thus having the potential to provide higher quantity and variety of seeds to the regenerating forest sites \citep{poorterSuccessionalTheories2023}. Similarly to forest age, larger forest cover tends to represent higher seed abundance and hence positively impacting the seed inflow to the regenerating site \citep{huancanunezSeedrainsuccessionalFeedbacksWet2021}. Furthermore, it is not only the presence of surrounding forest, but also the degree of connectivity between the existent forest patches that has a strong influence on the regenerating forest site. Well connected forest patches positively impact abundance of animal dispersed seeds, whereas more distanced forest patches have usually higher abundance of wind dispersed seeds \citep{dentUnitingNicheDifferentiation2021}. However, given that most seeds are distributed over short distances, sites in close proximity to an established forest tend to display higher seed abundance \citep{chazdonSecondGrowthPromise2014, hordijkLandUseHistory2024}.

The landscape factors such as the forest cover, the forest connectivity, and the forest age play a critical role in the seed abundance, diversity and distribution. Moreover, the surrounding forest functions as a seed source and as a habitat for seed dispersers and pollinators \citep{hordijkLandUseHistory2024, arroyo-rodriguezMultipleSuccessionalPathways2017}. Forests with larger cover tend to have higher abundance of seeds and thus having the potential to provide more seeds to the regenerating forest sites \citep{poorterSuccessionalTheories2023}. Furthermore, well connected forest patches positively impact abundance of animal dispersed seeds, whereas more distanced forest patches have usually higher abundance of wind dispersed seeds \citep{dentUnitingNicheDifferentiation2021}. However, given that most seeds are distributed over short distances, sites in close proximity to an established forest tend to display higher seed abundance \citep{chazdonSecondGrowthPromise2014, hordijkLandUseHistory2024}. Similarly to forest cover, older forest tends to represent higher seed abundance and hence positively impact the seed inflow to the regenerating site \citep{huancanunezSeedrainsuccessionalFeedbacksWet2021}. However, the dominant mode of dispersal might vary during different stages of succession. 

Early successional stages are most commonly characterized by wind, birds and bats dispersal, whereas later successional stages, when seed abundance increases are more commonly characterized by larger mammals dispersal \citep{dentUnitingNicheDifferentiation2021}. Moreover, dispersal mode can also vary between wet and dry tropical forest. In dry tropical forest, seeds are more likely to be smaller and dispersed by wind, whereas in wet tropical forest, seeds tend to be covered in fleshy fruit and dispersed by frugivores \citep{vieiraPrinciplesNaturalRegeneration2006, dentUnitingNicheDifferentiation2021}. Furthermore, unlike in dry tropical forest, succession in wet tropical forest is more dependent on the landscape connectivity and the availability of seed dispersers, as most tree species rely on animals for dispersal \citep{hordijkLandUseHistory2024, lohbeckSuccessionalChangesFunctional2013}.

%Landscapes
%at one extreme: landscape dominated by forested matrices are mainly determined %by local variation of environmental factors since seed abundance is relatively %high, whereas landscape dominated by various land use types will be mainly %determined by factors operating at landscape and regional scale, especially %those which affect production, dispersal, and predating of propagules

% gap
% While these mechanisms are broadly applicable, the factors shaping the successional pathway might vary substantially between ecosystems. For instance, in tropical forests, the process differs between dry and wet tropical forest. F wet tropical forest, succession is more dependent on the landscape connectivity and the availability of seed dispersers, as most tree species rely on animals for dispersal. In contrast, tropical dry forest is dominated by wind-dispersed species \citep{hordijkLandUseHistory2024, lohbeckSuccessionalChangesFunctional2013}. 
Despite the substantial research published on the mechanisms of secondary succession and to its related landscape factors, few studies have investigated the detailed effects of their impact on tree seed diversity and dispersal comparing dry and wet tropical forest ecosystems. Understanding the mechanisms of secondary succession broadens our knowledge on how ecosystems respond to a disturbance. This is particularly important in the age of Anthropocene during which global land use and climate change cause severe disruptions in the ecosystems \citep{poorterSuccessionalTheories2023}.  Therefore, the aim of this study is to understand how forest landscape factors affect seed diversity and dispersal in a dry and wet tropical forest during early succession. Given this, the following research questions have been identified: 

\textit{
How does forest age, cover and connectivity affect tree seed abundance, richness and dispersal mode in dry and wet secondary tropical forests during early succession?
\begin{itemize}
    \item  To what extent is the variation in seed abundance explained by forest age, forest cover, and forest connectivity in wet and dry tropical forests?
    \item To what extent is the variation in seed richness explained by forest age, forest cover, and forest connectivity in wet and dry tropical forests?
    \item To what extent is the variation in seed dispersal explained by forest age, forest cover, and forest connectivity in wet and dry tropical forests?
\end{itemize}
} 

By understanding the dynamics present in secondary succession, we can design and implement more effective ecosystem restoration strategies \citep{poorterSuccessionalTheories2023}.

%The terms investigated in this research are defined as following:\\\\
%\textit{Forest landscape factor terms.}\\
%\textbf{Forest age:} the number of years since a forest's establishment.\\
%\textbf{Forest cover:} spatial extent covered with trees.\\
%\textbf{Forest connectivity:} the degree of agglomerations between forest patches.\\
%\\
%\textit{Seed diversity and dispersal terms.} \\
%\textbf{Seed richness:} measure of the number of different seed species.\\
%\textbf{Seed evenness:} measure of the relative abundance of different seed species.\\
%\textbf{Seed dispersal mode:} the mechanisms by which a seed is moved from a source to an establishment site.\\

It is hypothesized that seed richness and evenness will increase with forest age in both dry and wet tropical forests (Fig. \ref{fig:cf}). Species richness will increase due to the arrival of new species over time and species evenness will increase due to decline in dominants pioneer species as early successional species are gradually replaced by later successional species \citep{poorterSuccessionalTheories2023, chazdonNaturalRegenerationTool2016}. Forest age is expected to have positive impact on biotic dispersal, however, negative impact on abiotic dispersal as abiotic dispersal is mostly present during early stages of succession \citep{dentUnitingNicheDifferentiation2021}. It is further hypothesized that forest connectivity and cover will have positive impact on seed richness and evenness as they facilitate larger and more varied seed pool. Moreover, it is hypothesized they have a negative impact on seed abiotic dispersal due to lower ability of wind dispersed seeds to travel further \citep{vieiraPrinciplesNaturalRegeneration2006}. In contrast, forest connectivity is hypothesized to have positive impact on biotically dispersed seeds as connectivity favors seed dispersers movement \citep{dentUnitingNicheDifferentiation2021}. 

% something that might be related to the age of the forest -> "This confirms that early-successional species invest in many small seeds that can travel large distances (e.g., by wind), whereas late-successional species invest in fruits that attract biotic dispersers to allow directional dispersal" (Lohbeck)

% Natural forest regeneration is an effective tool for various ecological restoration efforts such as climate change mitigation, biodiversity conservation, or soil health restoration \citep{hordijkLandUseHistory2024}. 
% It supports global ambitions such as the restoration of 3.5 million $km^2$ of degraded land \citep{holl2017restoring} and aligns with the UN Decade on Ecosystem Restoration, which aims to reach land degradation neutrality by 2030 \citep{waltham2020decade}. 

\begin{figure}[]
\begin{center}
		\includegraphics[]{Proposal/graphics/contextual_framework.drawio.png}
	\caption{Conceptual framework}
\label{fig:cf}
\end{center}
\end{figure}